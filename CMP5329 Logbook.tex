\documentclass[12pt]{report}
\usepackage{graphicx} % Required for inserting images
\usepackage[a4paper, margin=2.5cm]{geometry}
\graphicspath{{images/}}

\title{CMP5329 Logbook}
\author{Lewis Higgins - Student ID 22133848}
\date{January - March, 2024}


\usepackage[utf8]{inputenc}
\usepackage[T1]{fontenc}
\usepackage{float} % here for H placement parameter
\usepackage{subcaption}

\usepackage{filecontents}
\usepackage[
    firstinits=true, % render first and middle names as initials
    useprefix=true,
    maxcitenames=3,
    maxbibnames=99,
    style=authoryear,
    dashed=false, % re-print recurring author names in bibliography
    natbib=true,
    url=false
]{biblatex} % biblatex config for harvard refs

\renewbibmacro*{volume+number+eid}{%
    \printfield{volume}%
%  \setunit*{\adddot}% DELETED
    \setunit*{\addnbspace}% NEW (optional); there's also \addnbthinspace
    \printfield{number}%
    \setunit{\addcomma\space}%
    \printfield{eid}}
\DeclareFieldFormat[article]{number}{\mkbibparens{#1}}

\DeclareLabeldate{\field{date}\field{eventdate} \field{origdate}\literal{nodate}}

\addbibresource{logbook.bib}

% Use single quotes around titles:
\usepackage[british]{babel}
\usepackage{csquotes}

\usepackage{hyperref}

\hypersetup{
    colorlinks=true,
    linkcolor=black,
    filecolor=magenta,
    urlcolor=blue,
    citecolor=black,
}


\urlstyle{same}


% To prevent "Chapter N" display for each chapter
\usepackage[compact]{titlesec}
\usepackage{wasysym}
\usepackage{import}

\titlespacing*{\chapter}{0pt}{-2cm}{0.5cm}
\titleformat{\chapter}[display]
{\normalfont\bfseries}{}{0pt}{\Huge}

\newcommand\blfootnote[1]{
    \begingroup
    \renewcommand\thefootnote{}\footnote{#1}
    \addtocounter{footnote}{-1}
    \endgroup
}

\usepackage{fancyhdr}
\usepackage{calc}
\pagestyle{fancy}

\usepackage{tcolorbox}

\setlength\headheight{37pt}

\renewcommand{\chaptermark}[1]{%
    \markboth{#1}{}}

\lhead{Lewis Higgins - ID 22133848~~~~~~~~~~~~~~~\includegraphics[width=1.75cm]{bcu logo}}
\fancyhead[R]{\leftmark}

\begin{document}

    \makeatletter
    \begin{titlepage}
        \begin{center}
            \includegraphics[width=0.7\linewidth]{bcu logo}\\[4ex]
            {\huge \bfseries  \@title }\\[2ex]
            {\@author}\\[50ex]
            {\large \@date}
        \end{center}
    \end{titlepage}
    \makeatother
    \thispagestyle{empty}
    \newpage

    \tableofcontents
    %\footnotesize{\listoffigures}

    \chapter*{Introduction}\label{ch:introduction}
    \addcontentsline{toc}{chapter}{Introduction}

    This logbook documents the work completed and knowledge gained across the CMP5329 labs, showcasing
    the use of a wide variety of security techniques and access control methods on a Linux OS\@.
    This logbook specifically covers the following labs:
    \begin{itemize}
        \item Lab 1, covering OpenSSL\@.
        \item Lab 2, covering simple usage of GPG\@.
        \item Lab 5, covering the use of Linux Discretionary Access Control commands.
        \item Lab 6, covering password cracking.
    \end{itemize}

    \noindent As per module specifications, screenshots taken in each lab
    include the date and time at which they were taken.\\

    \vspace{50pt}

    \begin{tcolorbox}[colback=orange!5!white,colframe=orange!75!black,title=Example note]
        Additional notes, such as minor issues encountered or omitted screenshots due to
        work having already been done in earlier labs, are documented using these orange notes.
    \end{tcolorbox}

    \vspace{5pt}

    \begin{tcolorbox}[colback=red!5!white,colframe=red!75!black,title=Example important note]
        Critical issues that required special workarounds are documented using these red notes.
    \end{tcolorbox}

    \vspace{5pt}

    \begin{tcolorbox}[colback=pink!2!white,colframe=pink,title=Logbook navigation]
        This logbook was created in LaTeX, meaning that all references to figures within the
        logbook and references to external materials in the bibliography can be navigated by
        clicking on them.
        This also applies to the contents page.
    \end{tcolorbox}


    \chapter{OpenSSL}\label{ch:lab1}
    This lab was an introduction to the usage of OpenSSL to
encrypt and decrypt data using the DES and AES256 symmetric encryption algorithms,
as well as RSA private keys used in asymmetric encryption and how to generate and gather
public and private keys, alongside message digests.

\section{Version checking and ciphers}\label{sec:version}
To check the installed version of OpenSSL, "openssl version" can be executed.
The provided virtual machine from \href{https://moodle.bcu.ac.uk/mod/url/view.php?id=7914090}{the CMP5329 Moodle page}
uses OpenSSL version 1.1.1f, dated 31st March 2020.

\begin{figure}[H]
    \centering
    \includegraphics[width=.9\linewidth]{lab1/2}
    \caption{Getting the OpenSSL version}
    \label{fig:version}
\end{figure}

The list of OpenSSL ciphers can be viewed via "openssl ciphers".

\begin{figure}[H]
    \centering
    \includegraphics[width=.9\linewidth]{lab1/3}
    \caption{Getting the OpenSSL ciphers}
    \label{fig:ciphers}
\end{figure}

\section{Symmetric encryption}\label{sec:symmEncrypt}
Symmetric cryptography refers to the process of transferring data that has been encrypted by a single key.
Both the sender and receiver of this data use the same key to encrypt and decrypt the data.

\subsection{DES symmetric encryption}\label{subsec:des}
OpenSSL can be used to encrypt plaintext into ciphertext.
Many algorithms exist to generate ciphertext, but the DES symmetric encryption algorithm will be used here.

\begin{figure}[H]
    \centering
    \includegraphics[width=.9\linewidth]{lab1/4}
    \caption{Converting "a secret message" to ciphertext using DES with key "secretkey".}
    \label{fig:DESEncrypt}
\end{figure}

This ciphertext can then be decoded if you know the key it was encoded with.

\begin{figure}[H]
    \centering
    \includegraphics[width=.9\linewidth]{lab1/5}
    \caption{Decoding the ciphertext back to its original form using the key "secretkey".}
    \label{fig:DESDecrypt}
\end{figure}

\subsection{AES256 symmetric encryption and decryption}\label{subsec:aes256}
The DES algorithm is considered weak due to how simple it is to brute-force using today's processing power.
Newer algorithms were therefore developed, with one of these being AES\@.

I researched how to use this algorithm in OpenSSL, finding \href{https://www.madboa.com/geek/openssl/#how-do-i-simply-encrypt-a-file}{this help page}
\autocite{openSSLHelp} which provided details on encrypting text using the AES-256-cbc cipher.

\begin{figure}[H]
    \centering
    \includegraphics[width=.9\linewidth]{lab1/6}
    \caption{Encoding the plaintext with AES-256-cbc using the key "secretkey".}
    \label{fig:AESEncrypt}
\end{figure}

In this command the AES-256-cbc cipher is used, and the optional
-salt flag was added, which salts the text to provide different ciphertext.

Salting is the process of adding random data to the text prior to encoding it, which will change
the resulting ciphertext, making it harder to decrypt and increasing the strength of the encryption.

\pagebreak

\section{Asymmetric encryption}\label{sec:asymmEncryption}
Asymmetric cryptography is the practice of using two keys when transmitting data: a public key used to encrypt
data, and a private key used to decrypt it.
This is unlike symmetric encryption which uses one key for both users, but can be much more secure.
Data transferred this way has a digital signature attached, which allows for non-repudiation, as it cannot
be denied that the data originated from the user with the private key associated with the signature.
\newline

\subsection{Generating an RSA private key}\label{subsec:rsa-private-key}
OpenSSL can be used to generate these keys by using the "openssl genrsa" command.

\begin{figure}[H]
    \centering
    \includegraphics[width=.9\linewidth]{lab1/7}
    \caption{Generating a 2048-bit private key using genrsa.}
    \label{fig:genrsa}
\end{figure}

\pagebreak

\subsection{Storing DES3 \& passphrase encrypted RSA keys in a file}\label{subsec:storing-keys-in-file}
The key generated can then be encrypted using an encryption algorithm and a passphrase.
It can then be stored into a Privacy Enhanced Mail (.pem) file, which is
a file format 'to provide the creation and validation of digital signatures, and in addition the
encryption and decryption of signed data, based on asymmetric and symmetric cryptography.'
~\autocite[p. 1894]{PEMFormat}\\

\noindent In this example, a 1024-bit key is created using DES3 and the passphrase "secretkey".

\begin{figure}[H]
    \centering
    \includegraphics[width=.8\linewidth]{lab1/8}
    \caption{Generating and storing a 1024-bit private key using genrsa, DES3 and the passphrase "secretkey".}
    \label{fig:DES3Key}
\end{figure}

\begin{figure}[H]
    \centering
    \includegraphics[width=.8\linewidth]{lab1/9}
    \caption{The key stored in mykey.pem.}
    \label{fig:mykey}
\end{figure}

\subsection{Getting a public key from the private key}\label{subsec:PubFromPriv}
The private key stored into "mykey.pem" by the previous command can be accessed again to generate a public key.

\begin{figure}[H]
    \centering
    \includegraphics[width=.9\linewidth]{lab1/10}
    \caption{The public key generated from mykey.pem.}
    \label{fig:pubKey}
\end{figure}

\begin{figure}[H]
    \centering
    \includegraphics[width=.9\linewidth]{lab1/11}
    \caption{Storing the public key in a file.}
    \label{fig:pubKeyFile}
\end{figure}

\noindent "-out public" writes the key to a file called "public".
This file can then be read using cat.

\pagebreak

\subsection{Obtaining a message/file digest}\label{subsec:hashDigest}
To mitigate the risks of data interception or corruption, files can have "digests", which are the result
of hashing their contents.
If the file is modified whatsoever, the digest would be different.\newline

OpenSSL can generate digests using its "dgst" command.
\begin{figure}[H]
    \centering
    \includegraphics[width=.9\linewidth]{lab1/12}
    \caption{Creating a file, then getting the SHA1 digest of it.}
    \label{fig:digest}
\end{figure}

This can also be verified by using sha1sum, which returns the same digest.

\begin{figure}[H]
    \centering
    \includegraphics[width=.9\linewidth]{lab1/13}
    \caption{Verifying the digest.}
    \label{fig:digestVerify}
\end{figure}

\pagebreak

\subsection{Signing a digest}\label{subsec:SignDigest}
Signing a message digest using your private key definitively proves you sent it,
meaning that it cannot be denied that the file was sent, nor who it was sent by.

The previously used "example.txt" can again be used here to generate a digest encrypted using the "mykey.pem"
private key established earlier, which signs the digest.

\begin{figure}[H]
    \centering
    \includegraphics[width=.9\linewidth]{lab1/14}
    \caption{Writing a signed digest to a file.}
    \label{fig:writeDigest}
\end{figure}

Note that when we try to read this file, it is completely illegible, as it is not in Base64/ASCII format.
It can be converted to Base64 using OpenSSL's "enc" command.

\begin{figure}[H]
    \centering
    \includegraphics[width=.9\linewidth]{lab1/15}
    \caption{Encoding the signed digest to Base64.}
    \label{fig:base64Digest}
\end{figure}

Now that we have the signed digest, it can be verified using the public key, which confirms the authenticity
of the data in example.txt.

\begin{figure}[H]
    \centering
    \includegraphics[width=.9\linewidth]{lab1/16}
    \caption{Verifying the signature of example.txt.}
    \label{fig:signatureVerify}
\end{figure}

This returns "Verified OK" as intended.
If the file does get modified through either corruption or a threat agent's interference,
the digest would not be the same, seen below:

\begin{figure}[H]
    \centering
    \includegraphics[width=.9\linewidth]{lab1/17}
    \caption{Failing to verify the signature of example.txt, as it has been modified.}
    \label{fig:signatureVerifyFail}
\end{figure}


    \newpage

    \chapter{Usage of GPG}\label{ch:lab2}
    \begin{tcolorbox}[colback=red!5!white,colframe=red!75!black,title=Important note]
    In this lab and further labs using GPG, an incompatibility meant that instead
    of using the base GPG program, the alternative \textbf{GnuPG1} was used.
    Therefore, commands use the phrase "gpg1" instead of "gpg".
\end{tcolorbox}

This lab expanded on the concepts of asymmetric encryption through the use of\newline
GPG/GnuPG (GNU Privacy Guard) to produce, sign and verify public and private keys.\\

\section{Creating test users}\label{sec:testUsers}
For this lab, two test users were created and used to execute the necessary commands.

\subsection{Elevating the terminal}\label{subsec:sudo}
To add users to the system, administrative privileges are required.
To gain the necessary privileges, the command "sudo -s" or "sudo bash" can be entered
(both commands are functionally identical) which will change the terminal to be at root level.

\begin{figure}[H]
    \centering
    \includegraphics[width=.9\linewidth]{lab2/1}
    \caption{Elevating the terminal.}
    \label{fig:sudo}
\end{figure}

\subsection{Creating Bob and Alice}\label{subsec:createUsers}
With the elevated privileges gained from being a superuser, it is now possible to add users to the system using
"adduser" followed by the given username.
A password will then be necessary, followed by optional information such as phone numbers, which are left empty
for this lab.

\begin{figure}[H]
    \centering
    \includegraphics[width=.6\linewidth]{lab2/3}
    \caption{Creating user 'bob'}
    \label{fig:createBob}
\end{figure}

\begin{figure}[H]
    \centering
    \includegraphics[width=.6\linewidth]{lab2/2}
    \caption{Creating user 'alice'.}
    \label{fig:createAlice}
\end{figure}

For ease of access, multiple terminal tabs can be open at a time, so I elected to use one for the superuser root,
and one each for Bob and Alice.

\begin{figure}[H]
    \centering
    \includegraphics[width=.9\linewidth]{lab2/4}
    \caption{Multiple terminal tabs.}
    \label{fig:terminalTabs}
\end{figure}

I also added these new users to the "sudo" group, allowing them to also use the sudo command to execute commands
with elevated permissions.

\begin{figure}[H]
    \centering
    \includegraphics[width=.8\linewidth]{lab2/5}
    \caption{Adding bob and alice to sudo.}
    \label{fig:sudoAdd1}
\end{figure}

\begin{figure}[H]
    \centering
    \includegraphics[width=.8\linewidth]{lab2/5b}
    \caption{Adding bob and alice to sudo.}
    \label{fig:sudoAdd2}
\end{figure}

It is possible to switch the active terminal user using the command "su" followed by the account to switch to,
and then the password of the given account.

\begin{figure}[H]
    \centering
    \includegraphics[width=.9\linewidth]{lab2/5c}
    \caption{Switching the active terminal user. Note the prompt about running commands as an administrator,
    which signifies that they were successfully added to the sudo group.}
    \label{fig:suBobAlice}
\end{figure}

\pagebreak

\section{Exchanging encrypted files over an insecure channel}\label{sec:tmpExchange}
\begin{tcolorbox}[colback=red!5!white,colframe=red!75!black]
    For this section, assume that all commands have been executed on \textbf{both} the Bob and Alice
    user accounts unless stated otherwise.
\end{tcolorbox}
On standard Linux distributions, the /tmp directory is a public directory accessible to all users.
For this reason, it is therefore insecure, as every user on the system can read the files placed there.\footnote{However, they cannot update/change them without sudo permissions.}
To transfer files across insecure channels such as /tmp/, they should first be encrypted so that
they can only be read and/or used by their intended recipient.
Therefore, GNU Privacy Guard (GPG hereafter) can be used to generate and store public
and private asymmetric keys.

\subsection{Generating public/private key-pairs}\label{subsec:generating-private-keys}

To generate a private key, the command "gpg1 --gen-key" can be used.

\begin{figure}[H]
    \centering
    \includegraphics[width=.5\linewidth]{lab2/9}
    \caption{Generating a private key.}
    \label{fig:GPGgen}
\end{figure}

This will open a submenu where the user can select the kind of key they wish to generate,
as well as the size and expiry date of the key.
Once this is established, they must create a user ID if they didn't already have one,
consisting of their full name, email address and an optional comment.
While the key generates, the user is prompted to perform random inputs such as moving the mouse
and typing on the keyboard to enhance the randomness of the generated key.
A key was also generated for Alice.

\subsection{Exporting public keys}\label{subsec:exporting-public-keys}
It is possible to export the public keys from the generated key-pairs using GPG's export command.

\begin{figure}[H]
    \centering
    \includegraphics[width=.9\linewidth]{lab2/11b}
    \caption{Exporting Alice's public key.}
    \label{fig:GPGexport}
\end{figure}

\noindent This exports the public key in ASCII format (due to the use of the -a flag) to the file "alicepub.asc".
This can be seen by using "ls" to show the files in the directory.\footnote{The file can be read using "cat alicepub.asc", but it is a 2048-bit key, so it fills the terminal window.}
Because this is Alice's \textbf{public} key, we are comfortable sharing this to the public /tmp/ directory where all
users can see it.

\begin{figure}[H]
    \centering
    \includegraphics[width=.9\linewidth]{lab2/13}
    \caption{Copying Alice's public key to /tmp.}
    \label{fig:alicePubTmp}
\end{figure}


\subsection{Importing and signing public keys}\label{subsec:importing-public-keys}
Now that Alice's public key is in /tmp, Bob can copy this to his own directory and import it using GPG\@.

\begin{figure}[H]
    \centering
    \includegraphics[width=.9\linewidth]{lab2/14}
    \caption{Importing Alice's public key as Bob.}
    \label{fig:importAlice}
\end{figure}

\pagebreak

\noindent Bob can then \textbf{sign} this key, verifying that he trusts that this key does belong to Alice.
This is done by editing Alice's key as Bob, signing it, and then saving this.

\begin{figure}[H]
    \centering
    \includegraphics[width=.7\linewidth]{lab2/15}
    \caption{Bob signing Alice's public key.}
    \label{fig:signAliceKey}
\end{figure}

\pagebreak

\subsection{Encrypting and decrypting data}\label{subsec:encrDecr}
Now that Alice and Bob have their key-pairs generated, they can transfer asymmetrically encrypted data to each other.
This was tested by making a file, encrypting it using Alice's public key, and copying it to the /tmp directory.

\begin{figure}[H]
    \centering
    \includegraphics[width=.9\linewidth]{lab2/16}
    \caption{Making a file and encrypting it using Alice's public key.}
    \label{fig:encLab2Msg}
\end{figure}

This command can be broken down to its components:
\begin{itemize}
    \item -r alice - Sets Alice as the recipient of the file by using her public key.
    \item -o SECRET.asc - Outputs the encrypted data to SECRET.asc.
    \item -sea message.txt - Sign and encrypt the contents of message.txt in ASCII format.
\end{itemize}

\begin{figure}[H]
    \centering
    \includegraphics[width=.9\linewidth]{lab2/17}
    \caption{Copying the encrypted file to /tmp with the name "secret_to_alice.asc".}
    \label{fig:copyLab2Msg}
\end{figure}

Alice can then decrypt "secret\_to\_alice.asc" and output the results to "message.txt", where they can then
be read in human-legible form.

\begin{figure}[H]
    \centering
    \includegraphics[width=.9\linewidth]{lab2/18}
    \caption{Decrypting the encrypted message and reading it.}
    \label{fig:decryptLab2Msg}
\end{figure}


    \addtocounter{chapter}{2} % Makes the chapter number 5, so sections show as "5.1", "5.1.1"
    \chapter{Discretionary Access Control}\label{ch:lab5}
    This lab explored the use of Discretionary Access Control methods on a Linux system, which allows
the owner of an object to assign the level of access that others will have to said object.

\section{Creating test users and groups}\label{sec:creating-test-users}

\begin{tcolorbox}[colback=orange!5!white,colframe=orange!75!black,title=Note]
    Commands not pictured in this section were already showcased and explained in
    further detail in Lab 2, specifically in figures~\ref{fig:sudo}, \ref{fig:createBob}
    and \ref{fig:sudoAdd1} of section \ref{sec:testUsers}.
\end{tcolorbox}

\noindent For this lab, three test users "Pete", "Ali" and "Mary" were added to the system.
Pete and Ali were assigned to the "Boys" group, whereas Mary was assigned to the "Girls" group.

\subsection{Creating groups}\label{subsec:creating-groups}
With sudo privileges, additional groups can be added to the system.

\begin{figure}[H]
    \centering
    \includegraphics[width=.9\linewidth]{lab5/1}
    \caption{Making the "boys" and "girls" groups.}
    \label{fig:addgroup}
\end{figure}

\subsection{Adding users to groups}\label{subsec:adding-users-to-groups}

The new users were added to the groups mentioned above as well as the sudo group, enabling
their user accounts to run commands with sudo privileges.

\begin{figure}[H]
    \centering
    \includegraphics[width=.9\linewidth]{lab5/2}
    \caption{Adding the users to sudo and their respective groups.}
    \label{fig:addToGroup}
\end{figure}

We can verify which groups a given user is in by using the "groups [username]" command.

\begin{figure}[H]
    \centering
    \includegraphics[width=.9\linewidth]{lab5/3}
    \caption{Verifying that the users were added to the groups.}
    \label{fig:checkGroups}
\end{figure}

This can also be checked by viewing all groups on the system via "getent groups".

\begin{figure}[H]
    \centering
    \includegraphics[width=.9\linewidth]{lab5/3b}
    \caption{Seeing all groups (Boys and Girls are highlighted).}
    \label{fig:getent}
\end{figure}

\pagebreak

\section{Using chmod and chgrp to assign permissions}\label{sec:using-chmod}
Chmod is a command that changes the permissions for a given file or directory.
It can change permissions for reading, writing and executing files for the owner of the file,
a group of users, other users\footnote{Defined as users that aren't the owner or in an
associated group with permissions.} or all users.

\subsection{Mnemonic and octal chmod}\label{subsec:mnemonic-and-octal-chmod}
Chmod's arguments can be supplied as letters (mnemonic) or numbers (octal).
Functionally, there is no difference between the two, as the permissions will be changed regardless.
A mnemonic chmod command would be "chmod a \+rwx message.txt", which gives all users read, write
and execute permissions for the file message.txt.
An octal chmod command would be "chmod 777 message.txt", which gives the owner, group members
and others read, write and execute permissions.
The number 777 correlates to this because octal chmod operates by numeric sums.
Reading permissions are the number 4, writing is the number 2 and executing is the number 1.
Adding these together results in the number 7.

\subsection{Restricting directory access}\label{subsec:chgrp}
For the purposes of testing, a directory called D1 was added to user Mary's home.
This directory was associated with the girls group via chgrp, and modified with an octal chmod
command so that other users cannot access the directory whatsoever, but Mary herself and users
of the girls group can read and execute from it.

\begin{figure}[H]
    \centering
    \includegraphics[width=.9\linewidth]{lab5/4}
    \caption{Creating D1 and modifying its permissions.}
    \label{fig:D1}
\end{figure}

\noindent By using the command "ls -ld", the permissions of the directory are outputted.
The returned message reveals that:
\begin{itemize}
    \item The directory owner (Mary) has \textbf{R}ead, \textbf{W}rite and e\textbf{X}ecute permissions.
    \item Group members can \textbf{R}ead and e\textbf{X}ecute.
    \item Others can only e\textbf{X}ecute.
\end{itemize}

This can be tested using Ali and Pete's accounts, which are members of the boys group and not of the
girls group, meaning they are considered as "other users".

\begin{figure}[H]
    \centering
    \includegraphics[width=.8\linewidth]{lab5/5}
    \includegraphics[width=.8\linewidth]{lab5/6}
    \caption{Attempting to access D1 as Pete and Ali.}
    \label{fig:boysD1}
\end{figure}

\begin{tcolorbox}[colback=orange!5!white,colframe=orange!75!black,title=Note]
    An issue arose here that I didn't have another user in the girls group to test access with.
    To fix this, I went back and added the existing Alice account from Labs 1 and 2 to the girls group
    with \textit{sudo usermod -aG girls alice}.
\end{tcolorbox}

Using another account in the girls group, we can test if other girls can access the directory,
which succeeds.

\begin{figure}[H]
    \centering
    \includegraphics[width=.8\linewidth]{lab5/7}
    \caption{Successfully accessing D1 as Alice.}
    \label{fig:girlsD1}
\end{figure}

Because of the permissions set, Alice is permitted to read the d1 directory and execute files
within it, but she cannot write any files of her own, which is expected behaviour.

\begin{figure}[H]
    \centering
    \includegraphics[width=.8\linewidth]{lab5/8}
    \caption{Failing to write to D1 as Alice.}
    \label{fig:girlsWriteD1Fail}
\end{figure}

To test this further, the permissions can then be modified\footnote{770 is "rwxrwx---", which
correlates to the file owner and members of the group having read, write and execute permissions, but other users have none.}
again to allow girls to write files, which will then allow Alice to make the file.

\begin{figure}[H]
    \centering
    \includegraphics[width=.8\linewidth]{lab5/9}
    \includegraphics[width=.8\linewidth]{lab5/10}
    \caption{Modifying the permissions of D1, allowing Alice to write the file.}
    \label{fig:girlsWriteD1Success}
\end{figure}

\pagebreak

\subsection{Chgrp and chown}\label{subsec:using-chown}
Chown is a comment similar to chgrp, which assigns a file or directory's ownership
to a specific user.
For this example, we will create a directory in the shared /home folder and assign group ownership
to Boys via chgrp, and using chmod to allow all Boys all permissions, and all other users
read \& execute permissions.

\begin{figure}[H]
    \centering
    \includegraphics[width=.8\linewidth]{lab5/11}
    \caption{Making BoysDirectory and giving group ownership to Boys.}
    \label{fig:BoysDirectory}
\end{figure}

This new directory can be accessed by all users, but only written to by boys.
We can now test chown by making two subdirectories within BoysDirectory, where one will be owned
by Pete, and one by Ali.
Both users also modify the permissions of their own directories to only be accessible
by them.\footnote{700 is "rwx------", which means only the owner may read,
write and execute.}

\begin{figure}[H]
    \centering
    \includegraphics[width=.8\linewidth]{lab5/12}
    \includegraphics[width=.8\linewidth]{lab5/12b}
    \caption{Creating and modifying each user's directories and access permissions.}
    \label{fig:PeteAliDir}
\end{figure}

\begin{figure}[H]
    \centering
    \includegraphics[width=.8\linewidth]{lab5/12c}
    \includegraphics[width=.8\linewidth]{lab5/12d}
    \caption{Viewing the permissions of Pete and Ali's directories.}
    \label{fig:AliDirPerms}
\end{figure}

We can then prove that only the owners of the directories may access them by first accessing their
own directory, but then attempting to access the other user's directory:

\begin{figure}[H]
    \centering
    \includegraphics[width=.8\linewidth]{lab5/13}
    \includegraphics[width=.8\linewidth]{lab5/13b}
    \caption{Successfully accessing their own directory,
        but failing to access the other because the user isn't the owner.}
    \label{fig:PeteAliDirFail}
\end{figure}

An important distinction to be made here is that these subdirectories are owned by Pete and
Ali respectively.
They do \textbf{not} inherit the boys group ownership by default, meaning that commands to
change group permissions will be ineffective for other boys, as everyone who is not the owner
is considered "other" unless chgrp is used.\footnote{Just to demonstrate this, it was not used.}

\begin{figure}[H]
    \centering
    \includegraphics[width=.8\linewidth]{lab5/14}
    \includegraphics[width=.8\linewidth]{lab5/14b}
    \caption{Giving group RWX access to PeteDirectory}
    \label{fig:PeteDirFail}
\end{figure}

Ali still can't access the directory because he is not part of the "Pete" group, but he
(and any other users) would be able to use permissions given to "others".









    \chapter{Password Cracking}\label{ch:lab6}
    In this lab, a simple brute-force attack program written in C was used to crack a hashed account password.

\section{Linux password storage}\label{sec:linux-password-storage}
Linux systems store user account details across two files, $/etc/passwd$ and $/etc/shadow$.
I learned information about this from
\href{https://www.cyberciti.biz/faq/understanding-etcpasswd-file-format/}{this site}~\autocite{accStorage},
which states that the public unencrypted ASCII file $/etc/passwd$ contains a line for each user on the system,
with publicly accessible information such as username, user ID and group ID, whereas the encrypted $/etc/shadow$
file contains the encrypted passwords of users on the system.

\begin{figure}[H]
    \centering
    \includegraphics[width=.9\linewidth]{lab6/1}
    \includegraphics[width=.9\linewidth]{lab6/1b}
    \caption{Some of the contents of /etc/passwd, with the created users from earlier labs.}
    \label{fig:etcpasswd}
\end{figure}

\begin{figure}[H]
    \centering
    \includegraphics[width=.9\linewidth]{lab6/2}
    \includegraphics[width=.9\linewidth]{lab6/2b}
    \caption{Some of the contents of /etc/shadow, with the created users from earlier labs.}
    \label{fig:etcshadow}
\end{figure}



%\begin{itemize}
%    \item Username
%    \item Password - This doesn't store the actual password, which is located in $/etc/shadow$, but rather
%          an "x", indicative of if their encrypted and salted password is in the shadow file.
%    \item User ID
%    \item Primary group ID
%    \item User ID info - Comments such as their full name.
%    \item Home directory
%    \item The user's shell
%\end{itemize}



\section{crack.c}\label{sec:crack.c}
The provided code in crack.c is a small program that performs a dictionary attack,
cracking hashed passwords by hashing each word in the dictionary, adding the salt
and comparing the product to the hashed password.
The Ubuntu dictionary is located in $/usr/share/dict/words$.

\subsection{Importing and compilation}\label{subsec:importing-and-compilation}
First, this code must be ported into the Ubuntu VM using "nano crack.c", and pasting the code from Moodle.

\begin{figure}[H]
    \centering
    \includegraphics[width=.9\linewidth]{lab6/3}
    \caption{Porting crack.c into the VM.}
    \label{fig:nanoCrackC}
\end{figure}

\begin{figure}[H]
    \centering
    \includegraphics[width=.9\linewidth]{lab6/3b}
    \caption{Compiling crack.c.}
    \label{fig:compile}
\end{figure}

The '-lcrypt' argument supplies the Linux crypt library when compiling, allowing the crypt() function
to be used more securely.

\pagebreak

\subsection{Creating a test user}\label{subsec:creating-a-test-user}
To use the program, it will be necessary to create a new user, whose password can be found in the dictionary.
For this, the user 'fred' will be created, and his password will be 'peach'.
We can see if 'peach' appears in the dictionary, as well as the overall dictionary word count.

\begin{figure}[H]
    \centering
    \includegraphics[width=.9\linewidth]{lab6/4}
    \caption{Checking the dictionary for the word 'peach', which is the 71496th entry.}
    \label{fig:checkDict}
\end{figure}

\begin{figure}[H]
    \centering
    \includegraphics[width=.6\linewidth]{lab6/4b}
    \caption{Creating the 'fred' user, with the password 'peach'.}
    \label{fig:createFred}
\end{figure}

It will then be necessary to get the hashed version of Fred's password, which can be done using
"\textbf{cat /etc/shadow | grep fred | awk -F: '\{ print \$2 \}'}".
This command will read the shadow file, selecting the row starting with 'fred'.
Then, it will extract his hash by selecting the second column.

\begin{figure}[H]
    \centering
    \includegraphics[width=.9\linewidth]{lab6/5}
    \caption{Viewing fred's hashed password.}
    \label{fig:viewFredHash}
\end{figure}

\subsection{Cracking the password}\label{subsec:cracking-the-password}
The program takes two arguments, with the first being the salt used on the password and the
second being the entire hashed password.
It requires the salt as an argument because it will apply the salt to each password it checks.
As seen in Figure \ref{fig:viewFredHash}, Fred's entire hashed password is

\begin{verbatim}
$6$5vbOyjaMeVhIrG5x$WUkun/BiYW0Hcw.zX6m1K2Y7zQR0tVdLMIKDjK0rIDQmiNQfsZa
52n.qUo.x1eut6zoJzg3Sx0RJAavLZO2TN.
\end{verbatim}

We can figure out the salt used on this password by looking at how it starts.
The first 20 characters of this password are the salt, noticeable by how they are between two dollar signs.
This can be supplied as the first argument for the compiled crack program, and the second argument would be the
entire password, including the salt as well.
Ultimately, this forms the following command:

\begin{verbatim}
./crack '$6$5vbOyjaMeVhIrG5x$'
'$6$5vbOyjaMeVhIrG5x$WUkun/BiYW0Hcw.zX6m1K2Y7zQR0tVdLMIKDjK0rIDQmiNQfsZa
52n.qUo.x1eut6zoJzg3Sx0RJAavLZO2TN.'
\end{verbatim}

It is imperative to use \textbf{quotation} marks rather than speech marks, as Linux will otherwise incorrectly
interpret the arguments given due to there being dollar signs in the hash, meaning that the crack will be
unsuccessful.

\begin{figure}[H]
    \centering
    \includegraphics[width=\linewidth]{lab6/6}
    \caption{Entering the command.}
    \label{fig:cracking}
\end{figure}

\begin{figure}[H]
    \centering
    \includegraphics[width=.9\linewidth]{lab6/6b}
    \caption{The hashed password is revealed as 'peach'.}
    \label{fig:cracked}
\end{figure}

Note the differing timestamps, showing how this took around 3 minutes to execute, even on a 10-core
Intel i5-12600k.
This is because the word 'peach' occurs so late in the dictionary as seen in Figure \ref{fig:checkDict}.




    % "Suspicious formatting" - Not wrong, but this is only seen once in the document.
    \chapter*{Reflective report~~~~~~~~~~~\footnotesize{Cryptography and Access Control}}\label{ch:conclusion}
    \addcontentsline{toc}{chapter}{Reflective report on Cryptography and Access Control}
    \markboth{Reflective report}{}

    \small
    With the world moving forward into an increasingly digital age, the security of data is paramount for
    corporations, businesses and general end-users alike.
    Sensitive data such as bank details and important corporate documents being stored on digital servers inspires
    countless threat agents to gain unauthorised access to all kinds of devices with each passing year.
    In 2016, the Identity Theft Resource Center and Cyber Scout reported 1,093 data breach incidents, up 40\%
    from the 780 reported in 2015~\autocite{DataBreaches}.\\

    \noindent It is for this reason that many strategies are employed to secure digital systems, such as cryptography.
    Cryptography is defined as the technique of obfuscating or coding data~\autocite{KSCryptography}, and often refers
    to encryption in a cybersecurity context, wherein data transmitted and stored on systems
    is encrypted as an additional layer of security that ensures data cannot be read by anyone other than its
    intended recipient, even if it is intercepted during transmission or stolen.
    Asymmetric encryption is especially important, as it allows for non-repudiation, which provides
    assurance to the sender that its message was delivered, as well as proof of the sender's identity to the recipient.
    As a result, neither the sender nor the receiver can deny the message was sent and received~\autocite{NR}.\\

    \noindent While cryptography is essential in the security and integrity of data, it does not come
    without some disadvantages of its own, especially on portable and/or older, low-performance devices.
    Constant encryption and decryption of data using strong encryption algorithms such as AES256 can be
    performance-intensive, causing these devices to lag.
    Additionally, an unavoidable consequence of the security provided by strong encryption is that the data
    cannot be recovered without the key.
    Though this is intentional, there are likely to be scenarios wherein a user has lost their key and therefore
    all of their encrypted data, as cracking stronger encryption methods is not feasibly possible with
    current computational power in finite time~\autocite{AESFinite}.\\

    \noindent An additional cybersecurity measure utilised across a wide variety of systems is access control.
    One variant of this known as Discretionary Access Control (DAC) was showcased and explained in detail
    in Lab~\ref{ch:lab5} of this logbook.
    Access control is defined as "an essential element of security that determines who is allowed to
    access certain data, apps, and resources."~\autocite{AccessControlMS}.
    By limiting user access to only what is strictly necessary, risk can be significantly mitigated due to
    users being unable to modify or delete data they are not entitled to, intentionally or not.\\

    \noindent Access control is also not without some issues of its own.
    Access control is predicated on authentication, meaning that if a threat agent were to gain access to an account
    with superior access than their own via methods such as phishing or password cracking via tools like Hashcat or
    John the Ripper, they could then access privileged data via impersonation.
    Additionally, access control is reliant upon authorisation, meaning that if access levels are not correctly
    set by administrators, it is possible that users could gain access to data without having to gain malicious
    access at all.\\

    \noindent These two techniques are typically used in conjunction with each other on the vast majority of
    IT systems, which is known as Cryptographic Access Control.
    This is a vastly superior option to using just one of these techniques, as each technique amplifies the other.
    Cryptography mitigates access control's impersonation drawback because account passwords
    would be hashed and salted, making them much harder to brute force.
    It can also assist in role-based access control, assigning keys dependent upon a user's role
    and associated privilege level.
    With the application of cryptographic access control, digital systems can be much more secure and robust
    in the face of constantly evolving threats.

%    \large\textbf{You've not necessarily done what Lab 5 asks you to.
%    Yes, you've demonstrated the knowledge, but it's not in the specific way they wanted.
%    Redo it with the /home/photos folder. Could just mv it for rename, but you'll need to
%    redo your screenshots.}

    \printbibliography

\end{document}