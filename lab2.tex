\begin{tcolorbox}[colback=red!5!white,colframe=red!75!black,title=Important note]
    In this lab, an incompatibility meant that instead
    of using the base GPG program, the alternative \textbf{GnuPG1} was used.
    Therefore, commands use the phrase "gpg1" instead of "gpg".
\end{tcolorbox}

This lab expanded on the concepts of asymmetric encryption through the use of\newline
GPG/GnuPG (GNU Privacy Guard) to produce, sign and verify public and private keys.\\

\section{Creating test users}\label{sec:testUsers}
For this lab, two test users were created and used to execute the necessary commands.

\subsection{Elevating the terminal}\label{subsec:sudo}
To add users to the system, administrative privileges are required.
To gain the necessary privileges, the command "sudo -s" or "sudo bash" can be entered
which will change the terminal to be at root level.

\begin{figure}[H]
    \centering
    \includegraphics[width=.9\linewidth]{lab2/1}
    \caption{Elevating the terminal.}
    \label{fig:sudo}
\end{figure}

\subsection{Creating Bob and Alice}\label{subsec:createUsers}
Sudo allows us to add users to the system using "adduser" followed by the given username.
A password for the user will then be necessary, followed by optional information such as phone
numbers, which are left blank.

\begin{figure}[H]
    \centering
    \includegraphics[width=.49\linewidth]{lab2/2}
    \includegraphics[width=.49\linewidth]{lab2/3}
    \caption{Creating users 'bob' and 'alice'.}
    \label{fig:createBob}
\end{figure}

%\begin{figure}[H]
%    \centering
%    \includegraphics[width=.6\linewidth]{lab2/2}
%    \caption{Creating user 'alice'.}
%    \label{fig:createAlice}
%\end{figure}

For ease of access, multiple terminal tabs can be open at a time, so I elected to use one for the superuser root,
and one each for Bob and Alice.

\begin{figure}[H]
    \centering
    \includegraphics[width=.9\linewidth]{lab2/4}
    \caption{Multiple terminal tabs.}
    \label{fig:terminalTabs}
\end{figure}

I also added these new users to the "sudo" group, allowing them to also use the sudo command to execute commands
with elevated permissions.

\begin{figure}[H]
    \centering
    \includegraphics[width=.8\linewidth]{lab2/5b}
    \caption{Adding bob and alice to sudo.}
    \label{fig:sudoAdd1}
\end{figure}



It is possible to switch the active terminal user using the command "su" followed by the account to switch to,
and then the password of the given account.

\begin{figure}[H]
    \centering
    \includegraphics[width=.9\linewidth]{lab2/5c}
    \caption{Switching the active terminal user.}
    \label{fig:suBobAlice}
\end{figure}

\pagebreak

\section{Exchanging encrypted files over an insecure channel}\label{sec:tmpExchange}
\begin{tcolorbox}[colback=orange!5!white,colframe=orange!75!black]
    For this section, assume that all commands have been executed on \textbf{both} the Bob and Alice
    user accounts unless stated otherwise.
\end{tcolorbox}
On standard Linux distributions, the /tmp directory is a public directory where all users can read files placed there.
\footnote{However, they cannot update/change them without sudo permissions.}
To transfer files across insecure channels such as /tmp/, they should first be encrypted so that
they can only be read and/or used by their intended recipient.
Therefore, GPG can be used to generate and store public and private asymmetric keys.

\subsection{Generating public/private key-pairs}\label{subsec:generating-private-keys}

"gpg1 --gen-key" generates a private key.

\begin{figure}[H]
    \centering
    \includegraphics[width=.5\linewidth]{lab2/9}
    \caption{Generating a private key.}
    \label{fig:GPGgen}
\end{figure}

This will open a submenu where the user can select the kind of key they wish to generate,
and its size and expiry date
After this, they must create a user ID if one doesn't exist, with their full name,
email address and an optional comment.
While the key generates, the user is prompted to perform random inputs to enhance its entropy.
A key was also generated for Alice.

\pagebreak

\subsection{Exporting public keys}\label{subsec:exporting-public-keys}
It is possible to export the public keys from the generated key-pairs using GPG's export command.

\begin{figure}[H]
    \centering
    \includegraphics[width=.9\linewidth]{lab2/11b}
    \caption{Exporting Alice's public key.}
    \label{fig:GPGexport}
\end{figure}

\noindent This exports the public key in ASCII format (due to the use of the -a flag) to the file "alicepub.asc".
\footnote{The file can be read using "cat alicepub.asc", but it is a 2048-bit key, so it would completely fill the terminal window.}
Because this is Alice's \textbf{public} key, we are comfortable sharing this to the public /tmp/ directory where all
users can see it.

\begin{figure}[H]
    \centering
    \includegraphics[width=.9\linewidth]{lab2/13}
    \caption{Copying Alice's public key to /tmp.}
    \label{fig:alicePubTmp}
\end{figure}


\subsection{Importing and signing public keys}\label{subsec:importing-public-keys}
Bob can copy and import Alice's public key from /tmp\@.

\begin{figure}[H]
    \centering
    \includegraphics[width=.9\linewidth]{lab2/14}
    \caption{Importing Alice's public key as Bob.}
    \label{fig:importAlice}
\end{figure}

\pagebreak

\noindent Bob can then \textbf{sign} this key, verifying that he trusts that this key does belong to Alice.
This is done by editing Alice's key as Bob and signing it.

\begin{figure}[H]
    \centering
    \includegraphics[width=.7\linewidth]{lab2/15}
    \caption{Bob signing Alice's public key.}
    \label{fig:signAliceKey}
\end{figure}

\pagebreak

\subsection{Encrypting and decrypting data}\label{subsec:encrDecr}
Now that Alice and Bob have their key-pairs generated, they can transfer asymmetrically encrypted data to each other.
This was tested by making a file, encrypting it using Alice's public key, and copying it to the /tmp directory.

\begin{figure}[H]
    \centering
    \includegraphics[width=.9\linewidth]{lab2/16}
    \caption{Making a file and encrypting it using Alice's public key.}
    \label{fig:encLab2Msg}
\end{figure}

This command can be broken down to its components:
\begin{itemize}
    \item -r alice - Uses Alice's public key for encryption.
    \item -o SECRET.asc - Outputs the encrypted data to SECRET.asc.
    \item -sea message.txt - \textbf{S}ign and \textbf{e}ncrypt the contents of message.txt in \textbf{A}SCII format.
\end{itemize}

\begin{figure}[H]
    \centering
    \includegraphics[width=.9\linewidth]{lab2/17}
    \caption{Copying the encrypted file to /tmp with the name "secret\_to\_alice.asc".}
    \label{fig:copyLab2Msg}
\end{figure}

Alice can decrypt the file to "message.txt", where it can be read in human-legible form.

\begin{figure}[H]
    \centering
    \includegraphics[width=.9\linewidth]{lab2/18}
    \caption{Decrypting the encrypted message and reading it.}
    \label{fig:decryptLab2Msg}
\end{figure}
