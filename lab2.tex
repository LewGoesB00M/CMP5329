\begin{tcolorbox}[colback=red!5!white,colframe=red!75!black,title=Important note]
    In this lab and further labs using GPG, an incompatibility meant that instead
    of using the base GPG program, the alternative \textbf{GnuPG1} was used.
    Therefore, commands use the phrase "gpg1" instead of "gpg".
\end{tcolorbox}

This lab expanded on the concepts of asymmetric encryption through the use of\newline
GPG/GnuPG (GNU Privacy Guard) to produce, sign and verify public and private keys.\\

\section{Creating test users}\label{sec:testUsers}
For this lab, two test users were created and used to execute the necessary commands.

\subsection{Elevating the terminal}\label{subsec:sudo}
To add users to the system, administrative privileges are required.
To gain the necessary privileges, the command "sudo -s" or "sudo bash" can be entered
(both commands are functionally identical) which will change the terminal to be at root level.

\begin{figure}[H]
    \centering
    \includegraphics[width=.9\linewidth]{lab2/1}
    \caption{Elevating the terminal.}
    \label{fig:sudo}
\end{figure}

\subsection{Creating Bob and Alice}\label{subsec:createUsers}
With the elevated privileges gained from being a superuser, it is now possible to add users to the system using
"adduser" followed by the given username.
A password will then be necessary, followed by optional information such as phone numbers, which are left empty
for this lab.

\begin{figure}[H]
    \centering
    \includegraphics[width=.6\linewidth]{lab2/3}
    \caption{Creating user 'bob'}
    \label{fig:createBob}
\end{figure}

\begin{figure}[H]
    \centering
    \includegraphics[width=.6\linewidth]{lab2/2}
    \caption{Creating user 'alice'.}
    \label{fig:createAlice}
\end{figure}

For ease of access, multiple terminal tabs can be open at a time, so I elected to use one for the superuser root,
and one each for Bob and Alice.

\begin{figure}[H]
    \centering
    \includegraphics[width=.9\linewidth]{lab2/4}
    \caption{Multiple terminal tabs.}
    \label{fig:terminalTabs}
\end{figure}

I also added these new users to the "sudo" group, allowing them to also use the sudo command to execute commands
with elevated permissions.

\begin{figure}[H]
    \centering
    \includegraphics[width=.8\linewidth]{lab2/5}
    \caption{Adding bob and alice to sudo.}
    \label{fig:sudoAdd1}
\end{figure}

\begin{figure}[H]
    \centering
    \includegraphics[width=.8\linewidth]{lab2/5b}
    \caption{Adding bob and alice to sudo.}
    \label{fig:sudoAdd2}
\end{figure}

It is possible to switch the active terminal user using the command "su" followed by the account to switch to,
and then the password of the given account.

\begin{figure}[H]
    \centering
    \includegraphics[width=.9\linewidth]{lab2/5c}
    \caption{Switching the active terminal user. Note the prompt about running commands as an administrator,
    which signifies that they were successfully added to the sudo group.}
    \label{fig:suBobAlice}
\end{figure}

\pagebreak

\section{Exchanging encrypted files over an insecure channel}\label{sec:tmpExchange}
\begin{tcolorbox}[colback=blue!5!white,colframe=blue!75!black]
    For this section, assume that all commands have been executed on \textbf{both} the Bob and Alice
    user accounts unless stated otherwise.
\end{tcolorbox}
On standard Linux distributions, the /tmp directory is a public directory accessible to all users.
For this reason, it is therefore insecure, as every user on the system can read the files placed there.\footnote{However, they cannot update/change them without sudo permissions.}
To transfer files across insecure channels such as /tmp/, they should first be encrypted so that
they can only be read and/or used by their intended recipient.
Therefore, GNU Privacy Guard (GPG hereafter) can be used to generate and store public
and private asymmetric keys.

\subsection{Generating private keys}

To generate a private key, the command "gpg1 --gen-key" can be used.

\begin{figure}[H]
    \centering
    \includegraphics[width=.5\linewidth]{lab2/9}
    \caption{Generating a private key for Bob.}
    \label{fig:GPGgen}
\end{figure}

This will open a submenu where the user can select the kind of key they wish to generate,
as well as the size and expiry date of the key.
Once this is established, they must create a user ID if they didn't already have one,
consisting of their full name, email address and an optional comment.
While the key generates, the user is prompted to perform random inputs such as moving the mouse
and typing on the keyboard to enhance the randomness of the generated key.







